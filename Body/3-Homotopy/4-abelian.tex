\begin{frame}{Slight peeks of regularity} % .;

  \par Earlier, we saw $\pi_n (\mathbb{S}^2)$ was Abelian for $n \geq 2$.
  \vspace{\baselineskip}
  \par\attentive{Result:} For all $X \in \Top_{*}$ and $n \geq 2$, $\pi_n (X)$ is Abelian.
  \vspace{\baselineskip}
  \par Proof coming soon (rest of this talk)$\dots$
  
\end{frame}

\begin{frame}[t]{The higher homotopy groups are Abelian} % .;

  \par Fix $X \in \Top_{*}$ and $n \in \mathbb{Z}_{\geq 2}$.
  \par\attentive{Observation:} There are multiple group structures\footnote{To self: interchange} on $\pi_n (X)$
  
\end{frame}

\begin{frame}[t]{The higher homotopy groups are Abelian} % .;

  \par Fix $X \in \Top_{*}$ and $n \in \mathbb{Z}_{\geq 2}$.
  \par Our group structures fit together by
  \begin{align*}
    (a \mathbin{\textcolor{pink}{+_1}} b)
    \mathbin{\textcolor{cyan}{+_2}}
    (x \mathbin{\textcolor{pink}{+_1}} y)
    = (a \mathbin{\textcolor{cyan}{+_2}} x)
    \mathbin{\textcolor{pink}{+_1}}
    (b \mathbin{\textcolor{cyan}{+_2}} y)
  \end{align*}

  \vspace{2 \baselineskip}
  \par\attentive{Result:} Eckmann-Hilton

  \vspace{8 \baselineskip}
  \par\attentive{Corollary:} $\pi_n (X)$ is Abelian.
  
\end{frame}

\begin{frame}[t]{The Eckmann-Hilton argument}

  \par\attentive{Result:} Eckmann-Hilton. \textit{Suppose monoids $(A, \textcolor{pink}{\bullet}, \textcolor{pink}{\mathbf{1}})$ and $(A, \textcolor{cyan}{\circ}, \textcolor{cyan}{1})$ defined on the same set $A$ satisfy the \attentive{interchange law}}
  \begin{align*}
    (a \mathbin{\textcolor{pink}{\bullet}} b)
    \mathbin{\textcolor{cyan}{\circ}}
    (x \mathbin{\textcolor{pink}{\bullet}} y)
    = (a \mathbin{\textcolor{cyan}{\circ}} x)
    \mathbin{\textcolor{pink}{\bullet}}
    (b \mathbin{\textcolor{cyan}{\circ}} y)
  \end{align*}
  \par \textit{Then, $(\textcolor{pink}{\bullet}, \textcolor{pink}{\mathbf{1}}) = (\textcolor{cyan}{\circ}, \textcolor{cyan}{1})$ and the monoid is Abelian.}
  \vspace{0.25 \baselineskip}
  \par\attentive{Proof:} \only<1-1>{\footnotesize{(1/3)}}\only<2-2>{\footnotesize{(2/3)}}\only<3-3>{\footnotesize{(3/3)}}

\end{frame}
