\begin{frame}[t]{Duality: The opposite category} % .;

  \par\attentive{Definition:} Opposite category
  \vspace{5 \baselineskip}

  \par\attentive{Examples:}

  \vspace{7 \baselineskip}
  \par Working in $\C^{\mathrm{op}}$ rather than $\C$ is called \attentive{dualising}.
  
\end{frame}

\begin{frame}[t]{Duality: The logical view} % .;

  \begin{block}{Principle: Categorical duality}
    \par In any statement of the form ``for all categories $\C$, $\dots$'', we can replace $\C$ with $\C^{\mathrm{op}}$, and re-interpret the statement in $\C$.
  \end{block}

  \par\attentive{Example:} Initial and terminal objects
  
\end{frame}

\begin{frame}[t]{Duality: Examples} % .;

  \par\attentive{Definition:} Isomorphism. \textit{For all categories $\C$ and objects $x, y \in \C$, an \textbf{isomorphism} $x \simeq_{\C} y$ in $\C$ is$\dots$}
  \vspace{6\baselineskip}
  \par\attentive{Question:} What is the dual concept to isomorphisms?
  
\end{frame}

\begin{frame}[t]{Duality: Examples} % .;

  \par\attentive{Result:} Essential uniqueness of initial objects.
  \vspace{10\baselineskip}
  \par\attentive{Dual result:}
  
\end{frame}
