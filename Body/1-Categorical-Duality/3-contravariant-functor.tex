\begin{frame}[t]{Contravariant functors} % .;

  \par\attentive{Definition:} \textit{A \textbf{functor} $F : \C \to \D$ is a structure-preserving map}
  \begin{align*}
    \begin{tikzcd}[
      ampersand replacement = \;,
      cramped,
      sep = tiny
    ]
      \C \arrow[rr, "F"] \;         \; \D   \\
      x \arrow[dd, "f"'] \;         \; F x \arrow[dd, "F f"]  \\
                         \; \mapsto \;      \\
      y                  \;         \; F y
    \end{tikzcd}
    \qquad F(g \circ f) = F g \circ F f
    \qquad F 1_x = 1_{F x}
  \end{align*}
  \par\attentive{Dual definition:} Contravariant functor\footnote{To self: cod}.
  
\end{frame}

\begin{frame}[t]{Contravariant functors} % .;

  \par\attentive{Key example:} Representable functors. For $z \in \C$,

  % \begin{align*}
  %   \begin{tikzcd}[
  %     ampersand replacement = \;,
  %     cramped,
  %     sep = tiny
  %   ]
  %     \C  \arrow[rr, "{\C(z, \blank)}"] \;         \; \Set      \\
  %     x \arrow[dd, "f"']                \;         \; \C(z, x)  \\
  %                                       \; \mapsto \;           \\
  %     y                                 \;         \; \C(z, y)
  %   \end{tikzcd}
  %   \qquad \hspace{0.5 \linewidth}
  %   \\
  %   \begin{tikzcd}[
  %     ampersand replacement = \;,
  %     cramped,
  %     sep = tiny
  %   ]
  %     \C \arrow[rr, "{\C(\blank, z)}"]  \;         \; \Set^{\mathrm{op}}  \\
  %     x \arrow[dd, "f"']                \;         \; \C(x, z)            \\
  %                                       \; \mapsto \;                     \\
  %     y                                 \;         \; \C(y, z)
  %   \end{tikzcd}
  %   \qquad \hspace{0.5 \linewidth}
  % \end{align*}

\end{frame}

\begin{frame}[t]{Contravariant functors} % .;

  \par\attentive{Result:} \textit{\textbf{Functors preserve isomorphisms}. For all $\C \xrightarrow{F} \D$ and $f : x \simeq_{\C} y$, we have $F f : F x \simeq_{\D} F y$.}
  \vspace{10\baselineskip}
  \par\attentive{Dual result:}
  
\end{frame}

\begin{frame}[t]{Contravariant functors} % .;

  \par\attentive{Example application:} Isomorphic groups have isomorphic duals
  
\end{frame}
